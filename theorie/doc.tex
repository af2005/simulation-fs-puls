\documentclass[a4paper]{article}

%% Language and font encodings
\usepackage[utf8x]{inputenc}
\usepackage[T1]{fontenc}
\usepackage[ngerman]{babel}

%% Sets page size and margins
\usepackage[a4paper,top=3cm,bottom=2cm,left=3cm,right=3cm,marginparwidth=1.75cm]{geometry}
\setlength{\parindent}{0pt}

%% Useful packages
\usepackage{amsmath}
\usepackage{graphicx}
\usepackage[colorinlistoftodos]{todonotes}
\usepackage[colorlinks=true, allcolors=blue]{hyperref}

\title{Mathematischer Hintergrund und Variablenbenennung}
\author{Simon Jung, Bernd Lienau, Alexander Franke}
\date{\today}
\begin{document}
\maketitle

\section{Ebene Welle}
Wir betrachten zunächst eine einfache ebene Welle. Diese laufe in $z$-Richtung. Wir erhalten also für das elektrische Feld $E(z,t)$
\begin{align}
E(z,t) = E_0 \cdot \sin(\vec{k}z-\omega t)
\end{align}

Bzw. unter Verwendung der Frequenz $\nu$ und ohne den Wellenvektor $\vec{k} = \frac{\omega}{c}$

\begin{align}
E(z,t) &= E_0 \cdot \sin (\frac{2\pi\nu}{c} z - 2\pi\nu t ) \\
\Rightarrow E(z,t) &= E_0 \sin\left((2\pi\nu \left(\frac{z}{c} - t\right)\right)
\end{align}
 Für das menschliche Auge ist nur die Intensität des Lichts bedeutsam. Sie ist proportional zum Quadrat der
elektrischen Feldstärke der Lichtwelle: 
\begin{align}
I_\text{Licht} \propto E^2 
\end{align}
\section{Einzelspalt}
Wir definieren unseren Einzelspalt (und später unser Gitter) auf den Punkt $z=0$. Die Betrachtung des Beugungsmusters ist im Abstand $z_\text{Schirm}$ oder kurz $z_S$ dahinter. 
Unsere Spaltgröße ist definiert als $a$, ähnlich später unsere Gitterkonstante $a$. Wir verwenden Frauenhofer Beugung ($z_S \gg a$). Frauenhofer Beugung verlangt außerdem, dass eine ebene Welle auf den Spalt trifft.

Wir definieren als weitere Dimension für den Schirm die x-Achse ("nach oben und unten")  um konform mit einem Linkshand-Koordinatensystem zu sein. Die Maxima auf dem Schirm werden entsprechend $s_0, s_{+1}, s_{-1}, s_{+2}$ genannt.

Um das Wellenfeld beim Schirm $z_S$ zu erhalten, müssen wir nun
nach dem Huygens‘schen Prinzip alle Elementarwellen aus dem Spalt am Betrachtungsort
überlagern, d. h. die Summe der entsprechenden Felder bilden.
\begin{figure}[!htb]
\centering
\includegraphics[scale=0.8]{einzelspalt.pdf}
\caption{Skizze mit allen relevanten Variablen für den Einzelspalt}
\end{figure}

\section{1-dim N-Spalt)}
\end{document}